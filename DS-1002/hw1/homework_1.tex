\documentclass{article}

\usepackage{fancyhdr}
\usepackage{extramarks}
\usepackage{amsmath}
\usepackage{amsthm}
\usepackage{amsfonts}
\usepackage{tikz}
\usepackage[plain]{algorithm}
\usepackage{algpseudocode}
\usepackage{textgreek}

\usetikzlibrary{automata,positioning}

%
% Basic Document Settings
%

\topmargin=-0.45in
\evensidemargin=0in
\oddsidemargin=0in
\textwidth=6.5in
\textheight=9.0in
\headsep=0.25in

\linespread{1.1}

\pagestyle{fancy}
\lhead{\hmwkAuthorName}
\chead{\hmwkClass\ \hmwkTitle}
\rhead{\firstxmark}
\lfoot{\lastxmark}
\cfoot{\thepage}

\renewcommand\headrulewidth{0.4pt}
\renewcommand\footrulewidth{0.4pt}

\setlength\parindent{0pt}

%
% Create Problem Sections
%

\newcommand{\enterProblemHeader}[1]{
    \nobreak\extramarks{}{Problem \arabic{#1} continued on next page\ldots}\nobreak{}
    \nobreak\extramarks{Problem \arabic{#1} (continued)}{Problem \arabic{#1} continued on next page\ldots}\nobreak{}
}

\newcommand{\exitProblemHeader}[1]{
    \nobreak\extramarks{Problem \arabic{#1} (continued)}{Problem \arabic{#1} continued on next page\ldots}\nobreak{}
    \stepcounter{#1}
    \nobreak\extramarks{Problem \arabic{#1}}{}\nobreak{}
}

\setcounter{secnumdepth}{0}
\newcounter{partCounter}
\newcounter{homeworkProblemCounter}
\setcounter{homeworkProblemCounter}{1}
\nobreak\extramarks{Problem \arabic{homeworkProblemCounter}}{}\nobreak{}

%
% Homework Problem Environment
%
% This environment takes an optional argument. When given, it will adjust the
% problem counter. This is useful for when the problems given for your
% assignment aren't sequential. See the last 3 problems of this template for an
% example.
%
\newenvironment{homeworkProblem}[1][-1]{
    \ifnum#1>0
        \setcounter{homeworkProblemCounter}{#1}
    \fi
    \section{Problem \arabic{homeworkProblemCounter}}
    \setcounter{partCounter}{1}
    \enterProblemHeader{homeworkProblemCounter}
}{
    \exitProblemHeader{homeworkProblemCounter}
}

%
% Homework Details
%   - Title
%   - Due date
%   - Class
%   - Section/Time
%   - Instructor
%   - Author
%

\newcommand{\hmwkTitle}{Homework\ \#2}
\newcommand{\hmwkDueDate}{February 12, 2014}
\newcommand{\hmwkClass}{DS-1002}
\newcommand{\hmwkClassTime}{Section A}
\newcommand{\hmwkClassInstructor}{Professor Ling}
\newcommand{\hmwkAuthorName}{\textbf{Cody Fizette}}
\newcommand{\zed}{\mathbb{Z}}
\newcommand{\prob}{\mathbb{P}}

%
% Title Page
%

\title{
    \vspace{2in}
    \textmd{\textbf{\hmwkClass:\ \hmwkTitle}}\\
    \normalsize\vspace{0.1in}\small{Due\ on\ \hmwkDueDate}\\
    \vspace{0.1in}\large{\textit{\hmwkClassInstructor}}
    \vspace{3in}
}

\author{\hmwkAuthorName}
\date{}

\renewcommand{\part}[1]{\textbf{\large Part \Alph{partCounter}}\stepcounter{partCounter}\\}

%
% Various Helper Commands
%

% Useful for algorithms
\newcommand{\alg}[1]{\textsc{\bfseries \footnotesize #1}}

% For derivatives
\newcommand{\deriv}[1]{\frac{\mathrm{d}}{\mathrm{d}x} (#1)}

% For partial derivatives
\newcommand{\pderiv}[2]{\frac{\partial}{\partial #1} (#2)}

% Integral dx
\newcommand{\dx}{\mathrm{d}x}

% Alias for the Solution section header
\newcommand{\solution}{\textbf{\large Solution}}

% Probability commands: Expectation, Variance, Covariance, Bias
\newcommand{\E}{\mathrm{E}}
\newcommand{\Var}{\mathrm{Var}}
\newcommand{\Cov}{\mathrm{Cov}}
\newcommand{\Bias}{\mathrm{Bias}}

\begin{document}
\maketitle
\pagebreak

\begin{homeworkProblem}
	Justify the following results.
	\begin{enumerate}
	\item $A\cap \left(B\cup C\right)=\left(A\cap B\right)\cup \left(A\cap C\right)$
	\item $\left(\bigcup\limits_{i=1}^{\infty} A_i\right)^c=\bigcap\limits_{i=1}^{\infty} A_i^c$
	\\
	
	\end{enumerate}
	
	\textbf{Solution 1.1}
	
	To prove that $A\cap \left(B\cup C\right)=\left(A\cap B\right)\cup \left(A\cap C\right)$, we must show that:
	
	\[
		1.\;A\cap \left(B\cup C\right)\subseteq \left(A\cap B\right)\cup \left(A\cap C\right)
	\]
	\[
		2.\;\left(A\cap B\right)\cup \left(A\cap C\right)\subseteq A\cap \left(B\cup C\right)	
	\]
	
	Let $x\in A\cap \left(B\cup C\right)$.
	Then $x\in A$ and either $x\in B$ or $x\in C$.
	Without loss of generality assume that $x\in B$. Then $x\in A\cap B$. Since $A\cap B\subseteq \left(A\cap B\right)\cup \left(A\cap C\right)$, we 		conclude that $x\in \left(A\cap B\right)\cup \left(A\cap C\right)$, thus showing that $A\cap \left(B\cup C\right)\subseteq \left(A\cap B\right)			\cup \left(A\cap C\right)$.
	\\
	
	Next, let $y\in \left(A\cap B\right)\cup \left(A\cap C\right)\subseteq A\cap \left(B\cup C\right)$. Then either $y\in A$ and
	$y\in B$ or $y\in A$ and $y\in C$. Without loss of generality assume that $y\in A$ and $y\in B$. Then $y\in B\cup C$. 
	Thus $y\in A\cap \left(B\cup C\right)$. 
	So therefore $\left(A\cap B\right)\cup \left(A\cap C\right)\subseteq A\cap \left(B\cup C\right)$ 
	\\
	
	Since $A\cap \left(B\cup C\right)\subseteq \left(A\cap B\right)\cup \left(A\cap C\right)$ and $\left(A\cap B\right)\cup \left(A\cap C\right)\subseteq A\cap \left(B\cup C\right)$,
	we conclude that $A\cap \left(B\cup C\right)=\left(A\cap B\right)\cup \left(A\cap C\right)$.
	\\
	
	
	\textbf{Solution 1.2}
	
	Observe that
	
		\begin{align*}
			x\in \left(\bigcup\limits_{i=1}^{\infty} A_i\right)^c
			&\iff x\notin \bigcup\limits_{i=1}^{\infty} A_i && \text{By definition of set complement}
			\\
			&\iff x\notin A_i,\; \forall i\in \zed^+ && \text{By definition of set union}
			\\
			&\iff x\in A_i^c,\; \forall i\in \zed^+ && \text{By definition of set complement}
			\\
			&\iff x\in \bigcap\limits_{i=1}^{\infty} A_i^c && \text{By definition of set intersection}
		\end{align*}
		
	Thus $\left(\bigcup\limits_{i=1}^{\infty} A_i\right)^c=\bigcap\limits_{i=1}^{\infty} A_i^c$
	\\
				
\end{homeworkProblem}

%Problem 2
\begin{homeworkProblem}
	Given a sample space \textOmega \: with $n$ elements, compute the total number of distinct subsets of \textOmega .
	\\
	
	\textbf{Solution 2}
	\\
	
	When constructing a subset $A$, $\forall x_i \in \Omega$, either $x_i \in A$ or $x_i \notin A$. Since there are $n$ elements in 
	\textOmega \: and two possibilities for each element, by the Fundamental Counting Principle there are $2^n$ distinct subsets of \textOmega .

\end{homeworkProblem}

\pagebreak

%Problem 3
\begin{homeworkProblem}

\end{homeworkProblem}
	Two players, \textit{A} and \textit{B}, alternately and independently flip a coin and the first player to obtain a head wins. Asume player
	 \textit{A} flips first.
	 \\
	 
	 \begin{enumerate}
	 	\item If the coin is fair, what is the probability that \textit{A} wins?
	 	\item Suppose that $\prob \left(head\right)=p$. What is the probability that \textit{A} wins?
	 	\\
	 	
	 \end{enumerate}
	 
	 \textbf{Solution 3.1}
	 \\
	 
	 \textit{A} wins if the first head occurs on an odd numbered throw. Thus:
	 
	 \begin{align*}
	 	\prob \left(A\: wins\right)
	 	&= \sum_{i=0}^{\infty} \left(\frac{1}{2}\right)^{2i} \left(\frac{1}{2}\right)
	 	\\
	 	&= \frac{1}{2} \sum_{i=0}^{\infty} \left(\frac{1}{4}\right)^i
	 	\\
	 	&= \frac{1}{2} \left(\frac{1}{1-\frac{1}{4}}\right) && \text{Sum of geometric series}
	 	\\
	 	&= \frac{1}{2}\cdot \frac{4}{3} 
	 	\\
	 	&= \frac{2}{3}
	 \end{align*}
	 \\
	 
	 \textbf{Solution 3.2}
	 \\
	 
	 Let $q=1-p$
	 
	 \begin{align*}
	 	\prob \left(A\: wins\right)
	 	&= \sum_{i=0}^{\infty} \left(1-p\right)^{2i} \left(p\right)
	 	\\
	 	&= p \sum_{i=0}^{\infty} \left(q^2\right)^i
	 	\\
	 	&= \frac{p}{1-q^2} && \text{Sum of geometric series}
	 \end{align*}
	 
	 

\end{document}
